% Options for packages loaded elsewhere
\PassOptionsToPackage{unicode}{hyperref}
\PassOptionsToPackage{hyphens}{url}
%
\documentclass[
]{article}
\usepackage{amsmath,amssymb}
\usepackage{lmodern}
\usepackage{iftex}
\ifPDFTeX
  \usepackage[T1]{fontenc}
  \usepackage[utf8]{inputenc}
  \usepackage{textcomp} % provide euro and other symbols
\else % if luatex or xetex
  \usepackage{unicode-math}
  \defaultfontfeatures{Scale=MatchLowercase}
  \defaultfontfeatures[\rmfamily]{Ligatures=TeX,Scale=1}
\fi
% Use upquote if available, for straight quotes in verbatim environments
\IfFileExists{upquote.sty}{\usepackage{upquote}}{}
\IfFileExists{microtype.sty}{% use microtype if available
  \usepackage[]{microtype}
  \UseMicrotypeSet[protrusion]{basicmath} % disable protrusion for tt fonts
}{}
\makeatletter
\@ifundefined{KOMAClassName}{% if non-KOMA class
  \IfFileExists{parskip.sty}{%
    \usepackage{parskip}
  }{% else
    \setlength{\parindent}{0pt}
    \setlength{\parskip}{6pt plus 2pt minus 1pt}}
}{% if KOMA class
  \KOMAoptions{parskip=half}}
\makeatother
\usepackage{xcolor}
\IfFileExists{xurl.sty}{\usepackage{xurl}}{} % add URL line breaks if available
\IfFileExists{bookmark.sty}{\usepackage{bookmark}}{\usepackage{hyperref}}
\hypersetup{
  pdftitle={R Notebook},
  hidelinks,
  pdfcreator={LaTeX via pandoc}}
\urlstyle{same} % disable monospaced font for URLs
\usepackage[margin=1in]{geometry}
\usepackage{color}
\usepackage{fancyvrb}
\newcommand{\VerbBar}{|}
\newcommand{\VERB}{\Verb[commandchars=\\\{\}]}
\DefineVerbatimEnvironment{Highlighting}{Verbatim}{commandchars=\\\{\}}
% Add ',fontsize=\small' for more characters per line
\usepackage{framed}
\definecolor{shadecolor}{RGB}{248,248,248}
\newenvironment{Shaded}{\begin{snugshade}}{\end{snugshade}}
\newcommand{\AlertTok}[1]{\textcolor[rgb]{0.94,0.16,0.16}{#1}}
\newcommand{\AnnotationTok}[1]{\textcolor[rgb]{0.56,0.35,0.01}{\textbf{\textit{#1}}}}
\newcommand{\AttributeTok}[1]{\textcolor[rgb]{0.77,0.63,0.00}{#1}}
\newcommand{\BaseNTok}[1]{\textcolor[rgb]{0.00,0.00,0.81}{#1}}
\newcommand{\BuiltInTok}[1]{#1}
\newcommand{\CharTok}[1]{\textcolor[rgb]{0.31,0.60,0.02}{#1}}
\newcommand{\CommentTok}[1]{\textcolor[rgb]{0.56,0.35,0.01}{\textit{#1}}}
\newcommand{\CommentVarTok}[1]{\textcolor[rgb]{0.56,0.35,0.01}{\textbf{\textit{#1}}}}
\newcommand{\ConstantTok}[1]{\textcolor[rgb]{0.00,0.00,0.00}{#1}}
\newcommand{\ControlFlowTok}[1]{\textcolor[rgb]{0.13,0.29,0.53}{\textbf{#1}}}
\newcommand{\DataTypeTok}[1]{\textcolor[rgb]{0.13,0.29,0.53}{#1}}
\newcommand{\DecValTok}[1]{\textcolor[rgb]{0.00,0.00,0.81}{#1}}
\newcommand{\DocumentationTok}[1]{\textcolor[rgb]{0.56,0.35,0.01}{\textbf{\textit{#1}}}}
\newcommand{\ErrorTok}[1]{\textcolor[rgb]{0.64,0.00,0.00}{\textbf{#1}}}
\newcommand{\ExtensionTok}[1]{#1}
\newcommand{\FloatTok}[1]{\textcolor[rgb]{0.00,0.00,0.81}{#1}}
\newcommand{\FunctionTok}[1]{\textcolor[rgb]{0.00,0.00,0.00}{#1}}
\newcommand{\ImportTok}[1]{#1}
\newcommand{\InformationTok}[1]{\textcolor[rgb]{0.56,0.35,0.01}{\textbf{\textit{#1}}}}
\newcommand{\KeywordTok}[1]{\textcolor[rgb]{0.13,0.29,0.53}{\textbf{#1}}}
\newcommand{\NormalTok}[1]{#1}
\newcommand{\OperatorTok}[1]{\textcolor[rgb]{0.81,0.36,0.00}{\textbf{#1}}}
\newcommand{\OtherTok}[1]{\textcolor[rgb]{0.56,0.35,0.01}{#1}}
\newcommand{\PreprocessorTok}[1]{\textcolor[rgb]{0.56,0.35,0.01}{\textit{#1}}}
\newcommand{\RegionMarkerTok}[1]{#1}
\newcommand{\SpecialCharTok}[1]{\textcolor[rgb]{0.00,0.00,0.00}{#1}}
\newcommand{\SpecialStringTok}[1]{\textcolor[rgb]{0.31,0.60,0.02}{#1}}
\newcommand{\StringTok}[1]{\textcolor[rgb]{0.31,0.60,0.02}{#1}}
\newcommand{\VariableTok}[1]{\textcolor[rgb]{0.00,0.00,0.00}{#1}}
\newcommand{\VerbatimStringTok}[1]{\textcolor[rgb]{0.31,0.60,0.02}{#1}}
\newcommand{\WarningTok}[1]{\textcolor[rgb]{0.56,0.35,0.01}{\textbf{\textit{#1}}}}
\usepackage{graphicx}
\makeatletter
\def\maxwidth{\ifdim\Gin@nat@width>\linewidth\linewidth\else\Gin@nat@width\fi}
\def\maxheight{\ifdim\Gin@nat@height>\textheight\textheight\else\Gin@nat@height\fi}
\makeatother
% Scale images if necessary, so that they will not overflow the page
% margins by default, and it is still possible to overwrite the defaults
% using explicit options in \includegraphics[width, height, ...]{}
\setkeys{Gin}{width=\maxwidth,height=\maxheight,keepaspectratio}
% Set default figure placement to htbp
\makeatletter
\def\fps@figure{htbp}
\makeatother
\setlength{\emergencystretch}{3em} % prevent overfull lines
\providecommand{\tightlist}{%
  \setlength{\itemsep}{0pt}\setlength{\parskip}{0pt}}
\setcounter{secnumdepth}{-\maxdimen} % remove section numbering
\ifLuaTeX
  \usepackage{selnolig}  % disable illegal ligatures
\fi

\title{R Notebook}
\author{}
\date{\vspace{-2.5em}}

\begin{document}
\maketitle

\hypertarget{q3-d}{%
\section{------------------------ Q3 d)
--------------------------}\label{q3-d}}

\hypertarget{simulate-150-observations-for-three-ar2-models-with-differnt-phi-values}{%
\subsubsection{simulate 150 observations for three AR(2) models with
differnt phi
values:}\label{simulate-150-observations-for-three-ar2-models-with-differnt-phi-values}}

\hypertarget{a-phi10.6-phi20.3}{%
\subsubsection{a) phi1=0.6, phi2=0.3}\label{a-phi10.6-phi20.3}}

\hypertarget{b-phi1-0.4-phi20.5}{%
\subsubsection{b) phi1=-0.4, phi2=0.5}\label{b-phi1-0.4-phi20.5}}

\hypertarget{c-phi11.2-phi2-0.7}{%
\subsubsection{c) phi1=1.2, phi2=-0.7}\label{c-phi11.2-phi2-0.7}}

\hypertarget{hint-you-may-use-polyroot-function-in-r-to-find-roots-of-a-polynomial.}{%
\subsubsection{Hint: You may use polyroot function in R to find roots of
a
polynomial.}\label{hint-you-may-use-polyroot-function-in-r-to-find-roots-of-a-polynomial.}}

\hypertarget{d-simulate-each-the-models-in-a-c-with-150-observations-and-plot-theirs-series-and-their-sample-acf.}{%
\subsubsection{(d) Simulate each the models in (a)-(c) with 150
observations, and plot theirs series, and their sample
ACF.}\label{d-simulate-each-the-models-in-a-c-with-150-observations-and-plot-theirs-series-and-their-sample-acf.}}

\begin{Shaded}
\begin{Highlighting}[]
\CommentTok{\# For a) AR(2) with phi1=0.6, phi2=0.3}
\NormalTok{xc}\OtherTok{=}\FunctionTok{arima.sim}\NormalTok{(}\AttributeTok{n=}\DecValTok{150}\NormalTok{, }\FunctionTok{list}\NormalTok{(}\AttributeTok{ar=}\FunctionTok{c}\NormalTok{(}\FloatTok{0.6}\NormalTok{, }\FloatTok{0.3}\NormalTok{))) }
\CommentTok{\#simulated n = 150 values from this model and plotted the sample time series and the sample ACF for the simulated data.}
\NormalTok{x}\OtherTok{=}\NormalTok{xc}\SpecialCharTok{+}\DecValTok{10}               \CommentTok{\# adds 10 to make mean = 10. Simulation defaults to mean = 0.}
\FunctionTok{plot}\NormalTok{(x, }\AttributeTok{type=}\StringTok{"b"}\NormalTok{, }\AttributeTok{main =} \StringTok{"Simulated AR(2) Series with phi1=0.6, phi2=0.3 and n = 150 observation"}\NormalTok{) }
\end{Highlighting}
\end{Shaded}

\includegraphics{Assignment3_Q3_files/figure-latex/unnamed-chunk-1-1.pdf}

\begin{Shaded}
\begin{Highlighting}[]
\CommentTok{\#The above plot command plots lags versus the ACF values for lags 1 to 10. The ylab parameter labels the y{-}axis and the "main" parameter puts a title on the plot.}
\end{Highlighting}
\end{Shaded}

\begin{Shaded}
\begin{Highlighting}[]
\FunctionTok{acf}\NormalTok{(x, }\AttributeTok{xlim=}\FunctionTok{c}\NormalTok{(}\DecValTok{1}\NormalTok{,}\DecValTok{10}\NormalTok{), }\AttributeTok{main=}\StringTok{"ACF for simulated AR(2) Data with phi1=0.6, phi2=0.3 and n = 150 observation"}\NormalTok{)}
\end{Highlighting}
\end{Shaded}

\includegraphics{Assignment3_Q3_files/figure-latex/unnamed-chunk-2-1.pdf}

\begin{Shaded}
\begin{Highlighting}[]
\CommentTok{\# For a) AR(2) with phi1=0.6, phi2=0.3}
\NormalTok{rho1 }\OtherTok{\textless{}{-}} \FunctionTok{ARMAacf}\NormalTok{(}\AttributeTok{ar=}\FunctionTok{c}\NormalTok{(}\FloatTok{0.6}\NormalTok{,}\FloatTok{0.3}\NormalTok{), }\AttributeTok{ma =} \DecValTok{0}\NormalTok{, }\AttributeTok{lag.max =} \DecValTok{10}\NormalTok{) }\CommentTok{\# 10 lags of ACF for AR(2) with phi1 = 0.6 phi2 = 0.3}
\CommentTok{\# rho1 (ρ1) is the ACF solved by using R code}
\NormalTok{rho1}
\end{Highlighting}
\end{Shaded}

\begin{verbatim}
##         0         1         2         3         4         5         6         7 
## 1.0000000 0.8571429 0.8142857 0.7457143 0.6917143 0.6387429 0.5907600 0.5460789 
##         8         9        10 
## 0.5048753 0.4667488 0.4315119
\end{verbatim}

\begin{Shaded}
\begin{Highlighting}[]
\FunctionTok{plot}\NormalTok{(rho1, }\AttributeTok{type=}\StringTok{\textquotesingle{}h\textquotesingle{}}\NormalTok{, }\AttributeTok{xlab=}\StringTok{\textquotesingle{}h\textquotesingle{}}\NormalTok{, }\AttributeTok{main=}\StringTok{\textquotesingle{}ACF\textquotesingle{}}\NormalTok{)}
\FunctionTok{abline}\NormalTok{(}\AttributeTok{h=}\DecValTok{0}\NormalTok{)  }\CommentTok{\#adds a horizontal axis to the plot}
\end{Highlighting}
\end{Shaded}

\includegraphics{Assignment3_Q3_files/figure-latex/unnamed-chunk-4-1.pdf}

\begin{Shaded}
\begin{Highlighting}[]
\CommentTok{\# For b) AR(2) with phi1={-}0.4, phi2=0.5}
\NormalTok{rho1 }\OtherTok{\textless{}{-}} \FunctionTok{ARMAacf}\NormalTok{(}\AttributeTok{ar=}\FunctionTok{c}\NormalTok{(}\SpecialCharTok{{-}}\FloatTok{0.4}\NormalTok{,}\FloatTok{0.5}\NormalTok{), }\AttributeTok{ma =} \DecValTok{0}\NormalTok{, }\AttributeTok{lag.max =} \DecValTok{10}\NormalTok{) }\CommentTok{\# 10 lags of ACF for AR(2) with phi1 = 0.6 phi2 = 0.3}
\CommentTok{\# rho1 (ρ1) is the ACF solved by using R code}
\NormalTok{rho1}
\end{Highlighting}
\end{Shaded}

\begin{verbatim}
##          0          1          2          3          4          5          6 
##  1.0000000 -0.8000000  0.8200000 -0.7280000  0.7012000 -0.6444800  0.6083920 
##          7          8          9         10 
## -0.5655968  0.5304347 -0.4949723  0.4632063
\end{verbatim}

\begin{Shaded}
\begin{Highlighting}[]
\FunctionTok{plot}\NormalTok{(rho1, }\AttributeTok{type=}\StringTok{\textquotesingle{}h\textquotesingle{}}\NormalTok{, }\AttributeTok{xlab=}\StringTok{\textquotesingle{}h\textquotesingle{}}\NormalTok{, }\AttributeTok{main=}\StringTok{\textquotesingle{}ACF\textquotesingle{}}\NormalTok{)}
\FunctionTok{abline}\NormalTok{(}\AttributeTok{h=}\DecValTok{0}\NormalTok{)  }\CommentTok{\#adds a horizontal axis to the plot}
\end{Highlighting}
\end{Shaded}

\includegraphics{Assignment3_Q3_files/figure-latex/unnamed-chunk-6-1.pdf}

\begin{Shaded}
\begin{Highlighting}[]
\CommentTok{\# For c) AR(2) with phi1=1.2, phi2={-}0.7}
\NormalTok{rho1 }\OtherTok{\textless{}{-}} \FunctionTok{ARMAacf}\NormalTok{(}\AttributeTok{ar=}\FunctionTok{c}\NormalTok{(}\FloatTok{1.2}\NormalTok{,}\SpecialCharTok{{-}}\FloatTok{0.7}\NormalTok{), }\AttributeTok{ma =} \DecValTok{0}\NormalTok{, }\AttributeTok{lag.max =} \DecValTok{10}\NormalTok{) }\CommentTok{\# 10 lags of ACF for AR(2) with phi1 = 0.6 phi2 = 0.3}
\CommentTok{\# rho1 (ρ1) is the ACF solved by using R code}
\NormalTok{rho1}
\end{Highlighting}
\end{Shaded}

\begin{verbatim}
##           0           1           2           3           4           5 
##  1.00000000  0.70588235  0.14705882 -0.31764706 -0.48411765 -0.35858824 
##           6           7           8           9          10 
## -0.09142353  0.14130353  0.23356071  0.18136038  0.05413996
\end{verbatim}

\begin{Shaded}
\begin{Highlighting}[]
\FunctionTok{plot}\NormalTok{(rho1, }\AttributeTok{type=}\StringTok{\textquotesingle{}h\textquotesingle{}}\NormalTok{, }\AttributeTok{xlab=}\StringTok{\textquotesingle{}h\textquotesingle{}}\NormalTok{, }\AttributeTok{main=}\StringTok{\textquotesingle{}ACF\textquotesingle{}}\NormalTok{)}
\FunctionTok{abline}\NormalTok{(}\AttributeTok{h=}\DecValTok{0}\NormalTok{)  }\CommentTok{\#adds a horizontal axis to the plot}
\end{Highlighting}
\end{Shaded}

\includegraphics{Assignment3_Q3_files/figure-latex/unnamed-chunk-8-1.pdf}

\end{document}
